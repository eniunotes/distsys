\documentclass[CJK]{beamer}
\usepackage{CJKutf8}
\usetheme{Berkeley}


\title{视频会话管理关键技术研究}
\date{2012.1.11}
\author{学生: 牛 津\\导师: 倪 宏}
\institute[2011]{中科院声学所 网络新媒体}


\begin{document}
\begin{CJK}{UTF8}{gbsn}
\frame{\titlepage}

\section[Outline]{}
\frame{\tableofcontents}

\section{背景、意义及研究现状}

\subsection{背景}


\frame {
	\frametitle{背景}
	视频业务的迅猛发展
}

\subsection{意义}

\frame {
	\frametitle{意义}
	会话的建立是视频服务提供的前提,会话管理系统性能的好坏直接影响服务能否正常提供。
}

\subsection{研究现状}


\frame {
	\frametitle{研究现状}
	SIP作为语音业务的信令协议的相关研究:
	\begin{itemize}
	\item SIP服务器的性能评估
	\item SIP服务器的过载控制
	\item SIP服务器的动态扩展
	\end{itemize}
}

\section{研究内容}


\subsection{性能评估}

\frame {
	\frametitle{性能评估}
	\only<1->{
	内容
	\begin{itemize}
		\item 评估会话请求各处理步骤的时间开销所占比例
		\item 评估传输协议的选择对性能的影响
		\item 评估认证与否的选择对性能的影响
	\end{itemize}
	}
	\only<2->{
	意义
	\begin{itemize}
	\item 为系统的性能优化提供目标
	\item 为传输协议及认证与否的选择提供依据
	\end{itemize}
	}
	
}

\subsection{过载监测和控制}

\frame {
	\frametitle{过载监测和控制}
	\only<1->{
	内容
	\begin{itemize}
		\item 监测:根据各消息队列的实时情况估计系统的负载情况,以监测过载发生/将要发生
		\item 控制:协同的过载控制
	\end{itemize}
	}
	\only<2->{
	意义
	\begin{itemize}
	\item 减少过载情况下性能的下降
	\item 负载情况的估计是动态扩展的基础之一
	\end{itemize}
	}
	
}

\subsection{动态扩展}

\frame {
	\frametitle{动态扩展}
	\only<1->{
	内容
	\begin{itemize}
		\item 扩展控制器如何保证系统的可扩展性
	\end{itemize}
	}
	\only<2->{
	意义
	\begin{itemize}
		\item 简化系统维护
		\item 符合云的概念
	\end{itemize}
	}
	\only<3>{
	创新点
	\begin{itemize}
		\item 根据不同业务的负载情况动态调整负载均衡策略参数及阈值
	\end{itemize}
	}
}


\section{已完成工作}

\frame {
	\frametitle{已完成工作}
	\only<1>{
		\begin{itemize}
			\item<1->开放业务平台:会话管理服务器的调研和开发、跨域认证的调研、用户位置服务的设计、(参与)开放接口的设计和开放接口网关的设计和实现、(参与)交互服务器的设计、用户生成业务的调研、申请书中IMS和Parlay及Web2.0的调研、申请书中用户生成业务的调研和相关内容的撰写;
			\item<1->NGOSS:业务订购功能的升级;
			\item<1->融合通道网关:通道测试协议的设计。
		\end{itemize}
	}
}

\section{参考文献}

\frame {
	\frametitle{参考文献}
	\bibliographystyle{plain}
	\bibliography{library}
	\cite{Nahum2007}\cite{Shen2010}\cite{Vaquero2011}\cite{Kim2011}\cite{Wang2010}\cite{Ohta2009}\cite{Hilt2008}\cite{Shen2008}\cite{Sisalem2011a}
}
\end{CJK}
\end{document}